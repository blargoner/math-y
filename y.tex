% Explaining Y
% By John Peloquin
\documentclass[letterpaper,12pt]{article}
\usepackage{amsmath,fourier}

\newcommand{\Y}{\mathbf{Y}}
\newcommand{\om}{\mathbf{\omega}}
\newcommand{\Om}{\mathbf{\Omega}}
\newcommand{\Th}{\mathbf{\Theta}}

% Meta
\title{Explaining~\(\Y\)}
\author{John Peloquin}
\date{}

\begin{document}
\maketitle

\noindent In the lambda calculus, given a term~\(F\), how can we find a \emph{fixed point} for~\(F\)---that is, a term~\(X\) such that \(FX=X\)? Well, if infinite terms were allowed, we could just take
\[X=F(F(\cdots))\]
with \(F\)~repeated infinitely many times. Because then we would have
\[FX=F(F(F(\cdots)))=X\]
Sadly, infinite terms are not allowed. However, we can achieve the desired goal by constructing a term which ``generates'' infinitely many~\(F\)'s. First consider
\[\om=\lambda x.xx\]
and
\[\Om=\om\om\]
Then \(\Om\) \(\beta\)-reduces to itself infinitely since
\begin{align*}
\Om&=\om\om&&\text{by definition of }\Om\\
	&=(\lambda x.xx)\om&&\text{by definition of }\om\\
	&=\om\om&&\text{by }\beta\text{-reduction}\\
	&=\Om&&\\
	&=\cdots&&
\end{align*}
If we introduce \(F\) into~\(\om\), we can generate an~\(F\) at each \(\beta\)-reduction step of~\(\Om\). Let \(\om_F=\lambda x.F(xx)\) and \(\Om_F=\om_F\om_F\) (where \(x\)~does not occur free in~\(F\)). Then
\[\Om_F=\om_F\om_F=(\lambda x.F(xx))\om_F=F(\om_F\om_F)=F(\Om_F)\]
In other words, \(X=\Om_F\) is a fixed point of~\(F\).

We can generalize this and define the fixed-point combinator
\[\Y=\lambda f.\ \Om_f=\lambda f.(\lambda x.f(xx))(\lambda x.f(xx))\]
So \(\Y F=F(\Y F)\) is a fixed point of~\(F\) for any~\(F\). This is the \(\Y\)~combinator.

The \(\Y\)~combinator is important because it lets us define \emph{recursive} functions in the lambda calculus. For example, if we want~\(G\) such that \(GX=X(XG)\) for all~\(X\), we would have it if
\[G=\lambda x.x(xG)=(\lambda gx.x(xg))G\]
So we can just take \(G=\Y(\lambda gx.x(xg))\).

The \(\Y\)~combinator is not the only fixed-point combinator. Above, we might have interchanged the order of \(\lambda f\) and~\(\lambda x\) to obtain
\[A=\lambda xf.f(xxf)\]
and
\[\Th=AA\]
Then
\[\Th F=AAF=(\lambda f.f(AAf))F=F(AAF)=F(\Th F)\]
This is Turing's fixed-point combinator, which has certain advantages over the \(\Y\)~combinator. There are infinitely many more.
% References
\begin{thebibliography}{0}
\bibitem{barendregt} Barendregt, H. and E.~Barendsen. \textit{Introduction to Lambda Calculus.} 2000.
\end{thebibliography}
\end{document}
